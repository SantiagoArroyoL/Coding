\documentclass[11pt,letterpaper]{article}
\usepackage[margin=0.5in]{geometry}
\usepackage[T1]{fontenc}
\usepackage{amssymb}
\usepackage[utf8]{inputenc}
\usepackage{tikz}
\usepackage{forest}
\usetikzlibrary{shadows,arrows.meta}
\usepackage{multicol}
\DeclareUnicodeCharacter{2660}{\picas}
\DeclareUnicodeCharacter{2663}{\trebol}
\DeclareUnicodeCharacter{2665}{\heart}


\title{Tarea 1 | Estructuras Discretas}
\date{29-08-2019}
\author{Montiel Ledesma Edgar \\ Arriaga Camacho Vanessa Rubí }

\begin{document}

  \maketitle
  %\pagenumbering{gobble}%
  \pagenumbering{arabic}

  \section{Reglas de producción}
    \subsection{Cadena donde 4i+1 número de b's}
    S::= E | EC \\
    E::= EEEE | EEEEC \\
    F::= Ab | bA | AbA | b \\
    A::= aA | a | cA | c \\

    \subsection{Cadena donde no haya abc}
    S::= E \\
    E::= aA | bE | cE | b | c | ab$\bigtriangleup$ | a \\
    $\bigtriangleup$::= a | b | aA | bE \\

    \subsection{Expresión de tamaño 3i con i > 1}
    S::= EE \\
    E::= III | IIIE \\
    I::= a | b | c
  \section{Comprobar por árboles de derivación si las siguientes expresiones son validas}
  $\pagebreak$
  \section{Traducción de proposiciones a fórmulas de la lógica proposicional}
  \begin{multicols}{2}
    \subsection{ p $\rightarrow \neg$q}
          p: María fue al teatro \\
          q: Tener clases el martes
    \subsection{ p $\rightarrow$ q}
          p: Hay democracia \\
          q: No hay detenciones arbitrarias ni otras violaciones
    \subsection{ p $\vee$ q $\rightarrow \neg$r}
          p: Acepto este trabajo \\
          q: Dejo de pintar \\
          r: Realizaré mis sueños
    \subsection{1. (p $\vee$ q) $\rightarrow$ (r $\vee$ d) \\
    2. (r $\vee$ d) $\rightarrow \neg$s \\
    3. s $\rightarrow \neg$p \\
    ------------------------ \\
    $\neg$p}
          p: Tormenta Continúa \\
          q: Anochece \\
          r: Nos quedamos a cenar \\
          d: Nos quedamos a dormir \\
          s: Iremos mañana al concierto \\
  \end{multicols}
  \section{Indicar el conectivo lógico principal, su rango y mediante equivalencias lógicas reducirlo a su máxima expresión}
    \subsection{Conectivo principal y rango}
      \begin{multicols}{3}
        \begin{forest}
          [a) r $\leftrightarrow \neg$(p $\wedge \neg$q) $\rightarrow \neg$q
            [$\leftrightarrow$
              [r] [$\neg$(p $\wedge \neg$q) $\rightarrow \neg$q
                  [$\rightarrow$
                    [$\neg$(p $\wedge \neg$q)
                      [$\neg$
                      [p $\wedge$ q
                        [$\wedge$
                          [p]
                          [q]
                        ]
                      ]
                      ]
                    ]
                    [$\neg$q
                      [q]
                    ]
                  ]
                ]
            ]
          ]
        \end{forest}\\ \\
        \begin{forest}
        [b) p $\rightarrow \neg$r $\leftrightarrow \neg$p $\wedge$ q $\rightarrow \neg$q
          [$\leftrightarrow$
            [p $\rightarrow \neg$r
            [$\rightarrow$
              [p]
              [$\neg$r
                [$\neg$
                  [r]
                ]
              ]
            ]
            ]
            [$\neg$p $\wedge$ q $\rightarrow \neg$q
              [$\rightarrow$
                [$\neg$p $\wedge$ q
                  [$\wedge$
                    [$\neg$p
                      [$\neg$
                        [p]
                        ]
                    ]
                    [q]
                  ]
                ]
              ]
            ]
          ]
        ]
        \end{forest}
        \begin{forest}
          [c) $\neg$(p $\rightarrow$ q) $\rightarrow$ p $\vee$ q $\leftrightarrow \neg$p $\wedge$ q $\vee$ p
            [$\leftrightarrow$
              [$\neg$(p $\rightarrow$ q) $\rightarrow$ p $\vee$ q
                [$\rightarrow$
                  [$\neg$(p $\rightarrow$ q)
                    [$\neg$
                      [p $\rightarrow$ q
                        [$\rightarrow$
                          [p]
                          [q]
                        ]
                      ]
                    ]
                  ]
                  [p $\vee$ q
                    [p]
                    [q]
                  ]
                ]
              ]
            ]
            [$\neg$p $\wedge$ q $\vee$ p
            [$\vee$
              [$\neg$p $\wedge$ q
                [$\wedge$
                  [$\neg$p
                    [$\neg$
                      [p]
                    ]
                  ]
                  [q]
                ]
              ]
              [p]
            ]
            ]
          ]
        \end{forest}
      \end{multicols}
      \begin{forest}
        [p $\rightarrow$ r $\leftrightarrow$ s $\wedge \neg$(q $\rightarrow$ t $\vee$ s) $\rightarrow$ t $\wedge \neg$s
          [$\leftrightarrow$
            [p $\rightarrow$ r
              [$\rightarrow$
                [p]
                [r]
              ]
            ]
            [s $\wedge \neg$(q $\rightarrow$ t $\vee$ s) $\rightarrow$ t $\wedge \neg$s
            [$\rightarrow$
              [s $\wedge \neg$(q $\rightarrow$ t $\vee$ s)
                [$\wedge$
                  [s]
                  [$\neg$(q $\rightarrow$ t $\vee$ s)
                    [$\neg$
                      [q $\rightarrow$ t $\vee$ s
                        [$\rightarrow$
                          [q]
                          [t $\vee$ s
                          [$\vee$
                            [t]
                            [s]
                          ]
                          ]
                        ]
                      ]
                    ]
                  ]
                ]
              ]
            ]
            [t $\wedge \neg$s
              [$\wedge$
                [t]
                [$\neg$s
                  [$\neg$
                    [s]
                  ]
                ]
              ]
            ]
            ]
          ]
        ]
      \end{forest}
    \subsection{Equivalencias lógicas}
        a) r $\leftrightarrow \neg$(p $\wedge \neg$q) $\rightarrow \neg$q \\ \\
        %Otra línea
        \qquad [r $\wedge$ ($\neg$(p $\wedge \neg$q) $\rightarrow \neg$q)] $\vee$ [$\neg$r $\wedge$ ($\neg$($\neg$(p $\wedge \neg$q)) $\rightarrow \neg$q)] \\
        %Otra
        \qquad [r $\wedge$ ($\neg$($\neg$(p $\wedge \neg$q)) $\vee \neg$q)] $\vee$ [$\neg$r $\wedge$ ($\neg$($\neg$($\neg$(p $\wedge \neg$q)))$\wedge \neg$q)]\\
        %Otra
        \qquad [r $\wedge$ ($\neg$($\neg$p $\vee$ q) $\vee \neg$q)] $\vee$ [$\neg$r $\wedge$ ($\neg$($\neg$($\neg$p $\vee $q))$\wedge \neg$q)]\\
        %Otra
        \qquad [r $\wedge$ (p $\wedge \neg$ q) $\vee \neg$q] $\vee$ [$\neg$r $\wedge$ ($\neg$(p $\wedge \neg$q)$\wedge \neg$q)]\\
        %Otra
        \qquad [r $\wedge$ (p $\wedge \neg$ q) $\vee \neg$q] $\vee$ [$\neg$r $\wedge$ ($\neg$p $\vee$ q)$\wedge \neg$q]\\ \\
        %Otra
        --------------------------------------- \\
        \qquad \qquad p $\wedge \top$ $\vee$ p $\top$  \\ \\ \\
        %Empieza el b)
        \qquad b) p $\rightarrow \neg$r $\leftrightarrow \neg$p $\wedge$ q $\rightarrow \neg$q \\ \\
        %Otra
        \qquad $\neg$p $\vee \neg$r $\leftrightarrow \neg$($\neg$p $\vee$ q) $\vee \neg$q \\
        %Otra
        \qquad $\neg$p $\vee \neg$r $\leftrightarrow$ (p $\wedge \neg$q) $\vee \neg$q \\
        %Otra
        \qquad [$\neg$($\neg$p $\vee \neg$r) $\vee$ (p $\wedge \neg$q) $\vee \neg$q] $\wedge$ [($\neg$p $\vee \neg$r) $\wedge$ $\neg$((p $\wedge \neg$q) $\vee \neg$q)] \\ \\
        %Otra
        --------------------------------------- \\
        \qquad [(p $\wedge$ r) $\vee$ (p $\wedge \neg$q) $\vee \neg$q] $\wedge$ [($\neg$p $\vee \neg$r) $\wedge$ (p $\vee$ q) $\wedge$ q] \\ \\ \\ \\ \\
        %Empieza la c
        \qquad c) $\neg$(p $\rightarrow$ q) $\rightarrow$ p $\vee$ q $\leftrightarrow \neg$p $\wedge$ q $\vee$ p \\ \\
        %Otra
        \qquad $\neg$($\neg$p $\vee$ q) $\rightarrow$ p $\vee$ q $\leftrightarrow \neg$p $\wedge$ q $\vee$ p \\
        %Otra
        \qquad p $\wedge \neg$q $\rightarrow$ p $\vee$ q $\leftrightarrow \neg$p $\wedge$ q $\vee$ p \\
        %Otra
        \qquad $\neg$(p $\wedge \neg$q) $\vee$ (p $\vee$ q) $\leftrightarrow \neg$p $\wedge$ q $\vee$ p \\
        %Otra
        \qquad ($\neg$p $\vee$ q) $\vee$ (p $\vee$ q) $\leftrightarrow \neg$p $\wedge$ q $\vee$ p \\
        %Otra
        \qquad ($\neg$p $\vee$ q) $\vee$ (p $\vee$ q) $\leftrightarrow$ $\neg$p $\vee$ q $\wedge$ p \\
        %Otra
        \qquad ($\neg$p $\vee$ q) $\vee$ (p $\vee$ q) $\leftrightarrow$ ($\neg$p $\vee$ q) $\wedge$ p \\ \\
        %Otra
        \qquad r $\vee$ (p $\vee$ q) $\leftrightarrow$ r $\wedge$ p \\
        %Otra
        \qquad r $\vee$ (q $\vee$ p) $\leftrightarrow$ r $\wedge$ p \\
        %Otra
        \qquad (r $\vee$ q) $\vee$ p $\leftrightarrow$ r $\wedge$ p \\
        %Otra
        \qquad [$\neg$(r $\vee$ q) $\vee$ p] $\vee$ (r $\wedge$ p) $\wedge$ [((r $\vee$ q)$\vee$ p) $\vee \neg$(r $\wedge$ p)] \\ \\
        --------------------------------------- \\
        \qquad ($\neg$p $\vee$ q) $\wedge$ (p $\vee \neg$ q)\\ \\ \\
        %La última y nos vamos xd
        \qquad d) p $\rightarrow$ r $\leftrightarrow$ s $\wedge \neg$(q $\rightarrow$ t $\vee$ s) $\rightarrow$ t $\wedge \neg$s \\
        %Otra
        \qquad ($\neg$p $\vee$ r) $\leftrightarrow$ s $\wedge \neg$(q $\rightarrow$ t $\vee$ s) $\rightarrow$ t $\wedge \neg$s \\
        %Otra
        \qquad ($\neg$p $\vee$ r) $\leftrightarrow$ s $\wedge$ ($\neg$q $\vee$ t $\vee$ s) $\rightarrow$ t $\wedge \neg$s \\
        %Otra
        \qquad ($\neg$p $\vee$ r) $\leftrightarrow$ s $\wedge$ (q $\wedge \neg$ t $\wedge$ $\neg$s) $\vee$ t $\wedge \neg$s \\
        %Otra
        \qquad ($\neg$p $\vee$ r) $\leftrightarrow$ (s $\wedge \neg$q) $\vee$ (t $\wedge$ (s $\vee \neg$s) \\
        %Otra
        \qquad ($\neg$p $\vee$ r) $\leftrightarrow$ (s $\wedge \neg$q) $\vee$ (t $\wedge$ ($\bot$) \\
        %Otra
        \qquad $\neg$($\neg$p $\vee$ r) $\vee$ [(s $\wedge \neg$q) $\vee$ (t $\wedge$ ($\bot$)] $\wedge$  $\neg$($\neg$p $\vee$ r) $\vee$ $\neg$[(s $\wedge \neg$q) $\vee$ (t $\wedge$ ($\bot$)]\\ \\
        %OTraaaa
        --------------------------------------- \\
        \qquad (p $\wedge \neg$r) $\vee$ [(s $\wedge \neg$q) $\vee$ (t $\wedge$ ($\bot$)] $\wedge$  $\neg$($\neg$p $\vee$ r) $\vee$ [($\neg$s $\vee$ q) $\wedge$ ($\neg$t $\vee \top$)] \\
  \section{Traducir los argumentos lógicos a fórmulas de la lógica proposicional y demostrar si son correctos.}
  \begin{multicols}{3}

    \subsection{1. $\neg$p $\rightarrow$ (q $\vee$ r)\\
    2. s $\rightarrow$ t \\
    3. $\neg$r $\wedge$ s \\
    ------------------------ \\
    q}
    \subsection{1. p $\rightarrow$ (q $\vee$ r) \\
    2. s $\rightarrow \neg$r \\
    3. t $\rightarrow$ q \\
    4. s $\wedge$ t \\
    5. $\neg$ p \\
    ------------------------ \\
     u  }
    \subsection{1. p $\rightarrow$ q \\
    2. q $\rightarrow$ (r $\wedge$ s) \\
    3. $\neg$r $\vee \neg$s $\vee$ t \\
    4. p $\wedge$ s \\
    ------------------------ \\
     t}
   \end{multicols}
  \section{Por medio de Interpreraciones calcular los modelos válidos para los siguientes conjuntos de fórmulas:}
    \subsection{$\Gamma$ = $\{$ p $\rightarrow$ q $\vee$ r, $\neg$t $\rightarrow$ s, t $\rightarrow$ q, w $\wedge \neg$q, s $\rightarrow \neg$ $\}$ }
      \begin{multicols}{2}
        1 $\rightarrow$ q $\vee$ r = 1 \\
        2 $\neg$t $\rightarrow$ s = 1 \\
        3 t $\rightarrow$ q = 1 \\
        4 w $\wedge \neg$ q = 1 \\
        5 s $\rightarrow \neg$r = 1 \\
        6 $\mathcal{I}$ (w) = 1  \qquad por 4\\
        7 $\mathcal{I}$ ($\neg$q) = 1  \qquad por 4\\
        8 $\mathcal{I}$ (q) = 0   \qquad por 7, 4\\
        9 $\mathcal{I}$ ($\neg$t) = 0  \qquad por = 4, 7, 8, 3 \\
        10 $\mathcal{I}$ (t) = 0  \qquad por 9, 2\\
        11 $\mathcal{I}$ (s) = 1  \qquad por 10, 2\\
        12 $\mathcal{I}$ ($\neg$r) = 1  \quad por 5, 11\\
        13 $\mathcal{I}$ (r) = 0  \qquad por 12\\
        14 $\mathcal{I}$ (p) = 0 \qquad  por 1, 8,  13\\ \\ \\
        El único modelo válido que satisface el conjunto $\Gamma$ es: \\
        $\mathcal{I}$ (p) = 0 \\
        $\mathcal{I}$ (q) = 0\\
        $\mathcal{I}$ (r) = 0\\
        $\mathcal{I}$ (t) = 0\\
        $\mathcal{I}$ (s) = 1\\
        $\mathcal{I}$ (w) = 1\\ \\ \\ \\
      \end{multicols}
    \subsection{$\Gamma = \{$ r $\vee$ t, p $\rightarrow \neg$r $\vee$ s, t, $\neg$q $\rightarrow$ r, t $\rightarrow \neg$w, $\neg$r $\rightarrow$ w}
      \begin{multicols}{2}
        1 r $\vee$ t = 1 \\
        2 p $\rightarrow \neg$r $\vee$ s = 1 \\
        3 $\neg$q $\rightarrow$ r = 1 \\
        4 s $\rightarrow$ r = 1 \\
        5 t $\rightarrow \neg$w \\
        6 $\neg$r $\rightarrow$ w = 1 \\
        7 $\mathcal{I}$(t) = 1   \qquad por 3 \\
        8 $\mathcal{I}$($\neg$w) = 1   \qquad por 6, 8 \\
        9 $\mathcal{I}$(w) = 0   \qquad por 9 \\
        10 $\mathcal{I}$($\neg$r) = 1   \qquad por 7, 10 \\
        11 $\mathcal{I}$(s) = 0   \qquad por 11 \\
        12 $\mathcal{I}$(p) = 0   \qquad por 5, 11 \\
        13 $\mathcal{I}$(q) = 1/0   \qquad por 2, 11, 13 \\ \\ \\ \\
        Los dos modelos válidos que satisfacen el comjunto $\Gamma$ son los siguientes, en los cuales sólo cambia "q": \\
        $\mathcal{I}$ (r) = 1 \\
        $\mathcal{I}$ (t) = 1\\
        $\mathcal{I}$ (p) = 0\\
        $\mathcal{I}$ (s) = 0\\
        $\mathcal{I}$ (q) = 1/0\\
        $\mathcal{I}$ (w) = 1
      \end{multicols}
\end{document}
