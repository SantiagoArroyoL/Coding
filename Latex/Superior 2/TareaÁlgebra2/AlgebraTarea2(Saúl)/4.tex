\section{Un número $z > 1$ es compuesto si y sólo si tiene un divisor primo $p$ tal que $p^2 \leq z$}
    \begin{proof}
        $\rightarrow$ \\ \\
        Sea $z > 1$ es compuesto \\
        Por el Teorema Fundamental de la Aritmética sabemos que $z = \displaystyle \prod_{i = 1}^{\infty} p_i^{\alpha_i}$ \\
        Tomamos a $\min\{p_1,...,p_n\}= p_j$ \\
        Como $z$ es compuesto \\
        $\exists p_k$ tal que $p_k \neq p_j \vee \alpha_j > 1$ \\
        Si $\exists ! k$ tal que $p_k \neq p_j$ \\
        Como $p_i$ es el minimo 
        \begin{align}
            \notag p_j &< p_k \\
            \notag p_j^2 < p_kp_j &\leq \prod_{i = 1}^{\infty} p_i^{\alpha_i}
        \end{align}
        Si $\alpha_i > 1$
        \begin{align}
            \notag p_j^2 = p_j^2 &\leq \prod_{i = 1}^{\infty} p_i^{\alpha_i} \\
            \notag \therefore p_j^2 &\leq  \prod_{i = 1}^{\infty} p_i^{\alpha_i}
        \end{align}
        \\ $\leftarrow$ \\ \\
        $\exists q$ tal que $p|z \wedge p^2 \leq z$ \\
        Como $|p| > 0$ como p es primo \\
        $p \neq 1 \wdge p \neq -1 \rightarrow |p| > 1$ \\
        Si $|p| \neq z$ terminamos \\
        Supongamos que $p|z$ tal que $p \neq 1, p \neq -1 \wedge p \neq z $
        \begin{align}
            \tag{Como $z > 1$} p^2 &= z^2 \\
            \notag z^2 &> z \\
            \tag{Por transitividad} p^2 &> z
            \notag \therefore |p| &\neq z \wedge -p \neq z
        \end{align}
        Como $p \in \{1,-1,z,-z\} \wedge p|z$ es compuesto $\qquad \blacksquare$
    \end{proof}