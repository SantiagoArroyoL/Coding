\section{Supongamos que $a,b > 1$ pueden escribirse como $a = p_1^{\alpha_1}...p_n^{\alpha_n}$ y $ b =p_1^{\beta_1}...p_n^{\beta_n}$ con cada $n \in N$ y cada $p_i$ un entero primo.}
    \begin{align}
        \notag Sean \; a = \prod_{i=1}^{n} p_i^{\alpha^i}, \; a = \prod_{i=1}^{n} p_i^{\beta^i}
    \end{align}
    Tomamos en consideración entonces la propiedad demostrada en clase: 
    \begin{align}
        \tag{\MakeUppercase{\romannumeral 1}}a | b \leftrightarrow \alpha_i \leq \beta_i \forall i \in \mathbb{N}-\{0\}
    \end{align}
    \subsection{$(a;b) = p_1^{\gamma_1}...p_n^{\gamma_n}$ donde $\gamma_i = \min\{\alpha_i,\beta_i\} $}
        \begin{proof}
        Supongamos que $d = \displaystyle \prod_{i=1}^{n} p_i^{\gamma^i}$ donde $\gamma_i = \min\{\alpha_i,\beta_i\}$  tal que $d|a$ y $d|b$ \\
        P.D. $d$ es el m.c.d. \\
        Sea c = $\displaystyle \prod_{i=1}^{n} p_i^{\delta^i}$ tal que $c|a$ y $c|b$ \\
        Por \MakeUppercase{\romannumeral 1} Sabemos que $\delta_i \leq \alpha_i$ y $\delta_i \leq \beta$ \\
        Seguimos que $\delta_i \leq \min\{\alpha_i,\beta_i\}$, entonces $\delta \leq \gamma_i $, por lo que $c|d$, de esto podemos concluir que $d$ es el máximo común divisor \\
        Como demostramos que $d$ es el m.c.d. y puede escribirse como $\displaystyle \prod_{i=1}^{n} p_i^{\gamma^i}$ donde $\gamma = \min\{\alpha_i,\beta_i\}$ queda demostrada la igualdad $\qquad \blacksquare$
        \end{proof}
    \subsection{$[a;b]= p_1^{\gamma_1}...p_n^{\gamma_n}$ donde $\gamma_i = \min\{\alpha_i,\beta_i\}$}
        Supongamos que $d = \displaystyle \prod_{i=1}^{n} p_i^{\gamma^i}$ donde $\gamma = \max\{\alpha_i,\beta_i\}$  tal que $a|d$ y $b|d$ \\
        P.D. $d$  es el m.c.m. \\
        Sea $c = \displaystyle \prod_{i=1}^{n} p_i^{\delta^i}$ tal que $a|c$ y $b|c$ \\
        Por \MakeUppercase{\romannumeral 1} Sabemos que $\alpha_i\leq\delta_i$ y $\beta_i\leq\delta_i$ \\
        Seguimos que $\max\{\alpha_i,\beta_i\} \leq \delta_i$ entonces $\gamma_i \leq \delta_i$ por lo que $d$ es el mínimo común multiplo \\
        Como demostramos que $d$ es el m.c.m. y que puede escribirse de la forma $\displaystyle \displaystyle \prod_{i=1}^{n} p_i^{\gamma^i}$ donde $\gamma = \max\{\alpha_i,\beta_i\}$ queda demostrada la igualdad $\qquad \blacksquare$