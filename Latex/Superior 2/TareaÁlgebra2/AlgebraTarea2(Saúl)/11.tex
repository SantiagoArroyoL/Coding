\section{Sean $a,b,c,d,\in \mathbb{Z}$, entonces:}
    \subsection{$ax+by = b+c$ tiene solución si y sólo si $ax+by = c$ tiene solución.}
        $\rightarrow$ \\ \\
        Suponemos que $ax+by = b+c$ tiene solución ,es decir $\exists (a;b) \wedge (a;b)|b+c$ \\ 
        P.D. $(a;b)|c$ Porque sabemos que $(a;b) ya existe$ \\
        Como $(a;b)|b+c \rightarrow (a;b)|b$ y $(a;b)|c$ \\
        $\therefore (a;b)|c$ 
        \\ \\ $\leftarrow$ \\ \\
        Suponemos que $ax+by = c$ tiene solución, es decir $\exists (a;b) \wedge (a;b)|c$ \\
        P.D. $(a;b)|b+c$ 
        \begin{align}
            \notag ax + by &= c \\
            \notag ax +by + b &= c+b \\
            \notag ax +b(y + 1) &= b+c \\
            \notag \frac{ax}{(a;b)} + \frac{b(y+1)}{(a;b)} &= \frac{b+c}{(a;b)} \\
            \notag \therefore (a;b)|b+c
        \end{align}
        Como se cumple la ida y la vuelta, queda demostrado $\qquad \blacksquare$
    \subsection{$ax+by = c$ tiene solución si y sólo si $(a;b) = (a;b;c)$}
    \\ \\ $&\rightarrow$ \\ \\
    Como $ax +by = c$ tiene solución entonces $(a;b)|c$ \\
    Como $(a;b)|a$ y $(a;b)|b$, entonces $(a;b) = (a;b;c)$ 
    \\ \\ $\leftarrow$ \\ \\
    Sabemos que $(a;b) = (a;b;c)$ es decir $(a;b)|c$ pues $(a;b)t = c p$ para algún $t \in \mathbb{Z}$
        