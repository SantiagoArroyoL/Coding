\section{Sea $p \in \mathbb{N}$, entonces $p$ es primo si y sólo si $p \nmid (p−1)!$}
    \begin{proof}
        $\rightarrow$ \\ \\
        Como $p$ es primo, $p$ sólo divide a multiplos de $p$, es decir $p|pk$ con $k \in \mathbb{Z}$ \\
        Como todo multiplicando de $(p-1)!$ es positivo y menor a $p$ (Es decir, ninguno es igual a p) entonces $p\nmid (p-1)!$ \\
        \\ $\leftarrow$ \\ \\
        Como $p\nmid (p-1)!$ no existen dos multiplicandos en $(p-1)!$ que sean factores de $p$ pues si existieran $a,b$ factores de $p$ tal que $p=ab$ entonecs $p$ sería compuesto y sin pérdida de generalidad podríamos ver a $(p-1)!$ de la siguiente manera: 
        \begin{align}
            \notag (p-1)! &= 1\cdot ... \cdot a \cdot ... b \cdot ... \cdot (p-1) \\
            \tag{Como el producto es conmutativo} (p-1)! &= a\cdot b \cdot 1 \cdot   ... \cdot (p-1) \\
            \notag &= p \cdot 1 \cdot ... \cdot (p-1) \\
            \tag{Sabemos que $a|ab$} p|p(1\cdot ... \cdot (p-1)) &= (p-1)!
        \end{align}
        $\therefore p$ no es compuesto, $p$ es primo \\
        Notemos que suposimos $b \neq a$
        Si $a=b$ \\
        $p = a^2$, con a primo para que $p|(p-1)!$ \\
        $a^2|(a^2-1)!$ \\
        $\therefore$ deben existir por lo menos 2 números menores a $a$ multiplos de $2$ \\
        Sabemos que el $\min\{ka|ka>0\}$ son $a$ y $2a$
        \begin{align}
            \notag \therefore a^2 > a &\wedge a^2>2 \\
            \notag a > 1 &\wedge a >2 \\
            \notag \therefore a >& 2 \\
            \notag \therefore a^2 &> 4 \\
            \notag \therefore p &> 4
        \end{align}
        Como se cumple la ida y la vuelta queda demostrado $\qquad \blacksquare$
    \end{proof}