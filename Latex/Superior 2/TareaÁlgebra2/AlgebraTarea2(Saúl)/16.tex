\section{Sea $n\in \mathbb{Z}^+$, entonce n es impar si y solo si $1+2+3+...+(n-1) \stackrel{n}{\equiv} 0$}
\begin{proof}
    Si n es impar, n-1 es par,\\
    por suma de gauss $1+2+3+...+(n-1) =\frac{n(n-1)}{2}$\\
    Como  $n \stackrel{n}{ \equiv }0$\\
    $\therefore\frac{n(n-1)}{2}\stackrel{n}{ \equiv }0$
\end{proof}\\
\begin{proof}
    Si $ 1+2+3+...+n-1\stackrel{n}{ \equiv }0$\\
    por suma de gauss $1+2+3+...+(n-1) =\frac{n(n-1)}{2}$\\
    entonces  $\frac{n(n-1)}{2}\stackrel{n}{ \equiv }0$\\
    Si n es par, $\frac{n}{2}\in \mathbb{Z}^+$\\
    entonces $n-1\stackrel{n}{\equiv} 0$\\
    $\therefore n-1= kn$ con $k\in \mathbb{Z}$\\
    $\therefore 1= n(1-k)$\\
    $\therefore n = 1 \lor n = -1$ {\huge !}(n es par) \\ 
    y$\therefore$ n es impar\\
    
\end{proof}