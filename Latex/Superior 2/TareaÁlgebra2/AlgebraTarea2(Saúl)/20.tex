\section{Sea $a \in \mathbb{Z}$ tal que $(a;7) = 1$. Entonces:}
    \subsection{$a^6-1$ es divisible entre $7$}
        Como $(a;7) = 1$ notamos que $7 \nmid  a$ por que en caso contrario tendríamos $(a;7)=7$ \\
        Como $7$ es primo, por el Teorema de Fermat tenemos
        \begin{align}
            \notag a^7 &\stackrel{7}{\equiv} a \\
            \notag a^6 &\stackrel{7}{\equiv} 1 \\
            \tag{Por def. de congruencia} a^6- 1 &\in n\mathbb{Z}
        \end{align}
        Es decir, $7|a^6-1$
    \subsection{Para cualquier $n \in \mathbb{N} $, a $6n -1$ es divisible entre $7$.}
        Por Inducción sobre n \\
        \begin{proof}
            Caso Base $n=0$
            \begin{align}
                \notag a^6(0)-1 = 1-1 = 0 = 7(0)
            \end{align}
            Se cumple el caso base \\ \\
            Hipotesis de Inducción: $7|a^6-1 \forall n \in \mathbb{N}$ \\
            Paso Inductivo: $n+1$
            \begin{align}
                \notag a^{6(n+1)}-1 &\stackrel{7}{\equiv} 1 \\
                \notag a^{6n+6} &\stackrel{7}{\equiv} a^6 \\
                \notag a^{6(n+1)} &\stackrel{7}{\equiv} a^6 \\
            \end{align}
                Pero por la demostración de 20.1 sabemos que $a^6 &\stackrel{7}{\equiv} 1$
            \begin{align}
                \tag{Por transitividad} a^{6(n+1)} &\stackrel{7}{\equiv} 1 
            \end{align}
            Como se cumple el casbo base y el paso inductivo se cumple para toda $n \qquad \blacksquare$
        \end{proof}