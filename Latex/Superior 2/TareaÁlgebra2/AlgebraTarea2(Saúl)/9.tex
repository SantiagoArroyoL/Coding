\section{Si $a = p_1^{\alpha_1}...p_n^{\alpha_n}$ y $b = p_1^{\beta_1}...p_n^{\beta_n}$, entonces $(a;b) = 1$ si y sólo si, para cada $i \in \{1, ...,n\}$ se satisface que $\alpha_i = 0$ o $\beta_i = 0$}
    \begin{proof}
        $\rightarrow$ \\ \\
        Como $(a;b) = 1$ $a$ y $b$ no comparten nu factor multiplo de un primo en común. \\
        De manera que cuando $\alpha_i = 0$ ocurren dos casos
        \begin{enumerate}
            \item $\beta_i = 0$
            \item $\beta_i \neq 0$
        \end{enumerate}
        Si 1. tanto en $a$ como en $b$, $p_i^{\alpha_i} = p_i^{\beta_i} = p_i^0 = 1$ \\
        Si 2. en $a, p_i^0 = 1$; mientras que en $b, p_i^{\beta_i}$ es igual al multiplo de un primo correspondiente \\
        En ambos casos tenemos que $\alpha_i = 0$ ó $\beta_i=0$, que es lo que queremos demostrar
        \\ \\ $\leftarrow$ \\ \\
        Sabemos que si $\alpha_i = 0$ entonces $\beta_i \neq 0$ ó $\beta_i=0$ \\
        Es decir, cuando el indice $i$ es igual a $a$, por ejemplo $1$, pasa lo siguiente:
        \begin{align}
            \notag a = p_1^{\alpha_1} ... p_n^{\alpha_n} = 2^0 \cdot p_2^{\alpha_2} \cdot ... \cdot p_n^{\alpha_n}
            \notag b = p_1^{\beta_1} ... p_n^{\beta_n} = 2^0 \cdot p_2^{\beta_2} \cdot ... \cdot p_n^{\beta_n}
        \end{align}
        Lo mismo ocurre cuando $\beta_i = 0$ y $\alpha_i \neq 0$ ó $\alpha_i = 0$ \\
        De lo anterior notamos dos cosas:
        \begin{enumerate}
            \item $a$ y $b$ comparten el $1$ como factor común, cuando $\beta_i=0$ y $\alpha_i=0$
            \item Como nunca ocurre que $\alpha_i \neq 0$ y $\beta_i \neq 0$, $a$ y $b$ nunca tienen como multiplicandos al mismo multiplo de primos \\
        \end{enumerate}
        Es decir, $a$ y $b$ son primos relativos \\
            $\therefore (a;b) = 1$ \\
            Como se cumple la ida y la vuelta queda demostrado $\qquad \blacksquare$
    \end{proof} 
    