\section{Si $a,n \in \mathbb{Z}^+$ son tales que $a>1$ y $a^n+1$ es primo, entonces $a$ es par y $n$ es potencia de $2$}
    Si $a$ es impar 
    \begin{align}
        \notag a &\stackrel{2}{\equiv} 1 \\
        \notag a^n &\stackrel{2}{\equiv}1^n \\
        \notag a^n &\stackrel{2}{\equiv}1 \\
    \end{align}
    $\therefore a^n $ es impar \\
    $\therefore a^n +1$ es par y como ningún par es primo tenemos una contradicción \\
    Por 6.1 tenemos que $n=2^{m}x$ con $x$ impar y único
    \begin{align}
        \tag{Porque $x$ es impar} a^{2^{m}x}+1 = (a^{2^{m}x} - (-1)^x
    \end{align}
    Como $a-b|a^m-b^m$
    \begin{align}
        \notag a^{2^m}-(-1)&|a^{2^mx}-(-1)^x \\
        \tag{Como es primo} a^{2^m}-(-1) &= a^{2^mx}-(-1)^x \\
        \therefore &= a^2 +1 \qquad \blacksquare
    \end{align}