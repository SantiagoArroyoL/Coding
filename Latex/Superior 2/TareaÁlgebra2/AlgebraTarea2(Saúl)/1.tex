\section{}
    \subsection{El único primo positivo par es $2$.}
    Sea $p$ un primo positivo par
    \begin{align}
        \notag \rightarrow \exists k \in \mathbb{Z} \; tal \; que \; p &= 2k \\
        \tag{Como $2 > 0$ y $p > 0$} \rightarrow 2 = 1 \vee 2 = p
    \end{align}
    Como $2 \neq 1$, entonces $ 2 = p \qquad \blacksquare$
    \subsection{Los únicos primos positivos consecutivos son $2$ y $3$.}
        Sean $p$ y $(p+1)$ dos primos consecutivos \\
        Como $p$ es un primo positivo
        \begin{align}
            \notag &\rightarrow p \neq 0 \wedge p \neq 1 \wedge 0 \leq p \\
            \notag &\rightarrow p > 1
        \end{align}
        Por algoritmo de la división \\
        $\exists q,r (p = 2q+r, 0 \leq r < 2$ tal que $2q+r > 1$ \\
        Si $r = 0$\\
        $\quad p = 2q$, es decir es un primo positivo par
        \begin{align}
            \tag{Por 1.1} p = 2 \wedge p+1=3
        \end{align}
        Si $r=1$
        \begin{align}
            \notag p &= 2q+1 \\
            \notag (p+1) &= 2q+2 \\
            \tag{Es decir un primo positivo par} (p+1) &= 2(q+1) \\
            \notag \therefore p+1 &= 2 \\
            \notag &\rightarrow p= 1 \qquad !
        \end{align}
        Llegamos a una contradicción ya que 1 no es primo $\qquad \blacksquare$
    \subsection{Los únicos tres impares positivos consecutivos que son primos son $3, 5, 7$}
        Sean $p,p+2,p+4$ primos positivos consecutivos impares\\
        Por algoritmo de la división: \\
        \begin{align}
            \notag \exists q,r \in \mathbb{Z} (p = 3q+r, 0 \leq r < |3| = 3)
        \end{align}
        Si $r=0$ 
        \begin{align}
            \tag{Es decir $3|p$} p &= 3q \\
            \tag{Como $3 \neq 1$} p &= 2 \\
            \notag \therefore p+2 = 5 &\wedge p+4 = 7
        \end{align}
        Si $r=1$
        \begin{align}
            \notag p &= 3q+1 \\
            \notag p+2 &= 3q+3 \\
            \tag{Es decir $3|p+2$} p+2 &= 3(q+1) \\
            \tag{Como $3 \neq 1$} p+2 &= 3 \\
            \notag \rightarrow& p = 1 !
        \end{align}
        Si $r=2$
        \begin{align}
            \notag p &= 3q+2 \\
            \notag p+4 &= 3q+6 \\
            \tag{Es decir $3|p+4$} p+4 &= 3(q+2) \\
            \tag{Como $ 3 \neq 1$} p+4 = 3 \\
            \notag \rightarrow& p = -1 ! \qquad \blacksquare
        \end{align}