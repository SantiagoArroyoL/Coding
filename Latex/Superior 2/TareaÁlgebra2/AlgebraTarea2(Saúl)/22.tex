\section{Sea $p>1$, entonces p es primo si y solo si $p| (m-1)! +1$}
\subsection{Si p es primo entonces  $p| (p-a)! +1$}
\begin{proof}
    Si p es primo por el teorema de wilson 
    \begin{align}
        \notag (p-1)! \stackrel{p}{\equiv} -1 \\
        \notag (p-1)!+1 \stackrel{p}{\equiv} 0
    \end{align}
     $\therefore p| (p-1)! +1 $
\end{proof}

\subsection{Si $p| (p-a)! +1$ entonces p es primo}
\begin{proof}
    Si $p| (p-a)! +1$ \\
    \begin{align}    
        \notag \exists  k \in \mathbb{Z}(pK = (p-1)! +1) \\
    \end{align}
    Supongamos $p|(p-1)!$
    \begin{align}
        \notag \exists  h \in \mathbb{Z}(ph = (p-1)!) \\
        \notag ph+1=(p-1)!+1 \\
        \notag ph+1=pk con h, k \in \mathbb{Z} \\
        \notag 1 = pk-ph \\
        \notag 1 = p(k-h)
    \end{align}
    $\therefore (p=1\land k-h =1) \lor (p=-1\land k-h =-1)$ {\huge !}(Contradiccion  con $p>1$)\\
    $\therefore p \nmid (p-1)!$\\
    y por (5) p es primo 
\end{proof}