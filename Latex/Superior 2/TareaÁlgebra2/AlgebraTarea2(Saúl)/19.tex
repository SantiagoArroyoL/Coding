\section{\large La produccion diaria de huevos en una granja es inferior a 75. Cierto dia, el recolector informa que la cantidad de huevos recogida es tal que contados de tres en tres sobran dos huevos, contados de cinco en cinco sobran 4 y contados de siete en siete sobran 5}

\begin{proof}
    Nuestro sistema a resolver es:\\
    \begin{align}
        \notag x \stackrel{3}{\equiv} 2\\
        \notag x \stackrel{5}{\equiv} 4\\
        \notag x \stackrel{7}{\equiv} 5\\
    \end{align}
    Por lo que se puede resolver\\ \\
    
    Resolviendo:\\
    \begin{align}
        \notag x \stackrel{7}{\equiv} 5\\
        \notag x \stackrel{5}{\equiv} 4\\
    \end{align}
    Llegamos a la ecuacion lineal: $7y\stackrel{5}{\equiv}4-5$\\
Reduciendo: 7y=-1(mod 5)\\
Pasandolo a su forma diofantina: 7y+5x=-\\
Por algoritmo de euclides: \\
5=7(0)  +5\\
7=5(1)  +2\\
5=2(2)  +1\\
2=1(2)  +0\\
(7,5)=1\\
Expresandolo como una combinacion lineal:\\
1=(5)(1)+(2)(-2)\\
1=(5)(3)+(7)(-2)\\
1=(5)(3)+(7)(-2)\\
(y0', z0')=(3,-2)\\
Y asi la solucion del sistema es (y0, z0)=(-3,2)\\
Por lo que y0= -3\\
De forma que y =-3+ (5/1)z\\
y =-3+5z\\
Entonces x =5+7(-3+5z)\\
Pasandolo a congruencias: x=-16(mod35)\\
Finalmente nuestro sistema de ecuaciones es:\\
x = 2(mod3)\\
x = -16(mod35)\\


Resolviendo:\\ \ 
x = -16(mod35)\\
x = 2(mod3)\\
Llegamos a la ecuacion lineal: 35y=2--16(mod 3)\\
Reduciendo: 35y=18(mod 3)\\
Pasandolo a su forma diofantina: 35y+3x=0\\
Por lo que y0= 0\\
De forma que y =0+ (3/1)z\\
y =0+3z\\
Entonces x =-16+35(0+3z)\\
Pasandolo a congruencias: x=-16(mod105)\\
Finalmente nuestro sistema de ecuaciones es:\\
x = -16(mod105)\\
x = 89(mod105)\\
Por lo que no se puede realizar con esas condiciones, el capataz tiene razon

\end{proof}