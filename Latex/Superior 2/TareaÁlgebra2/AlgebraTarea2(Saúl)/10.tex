\section{. Sean $a,b \in \mathbb{Z}^+$ tales que $(a;b) =1$. Si $ab= z^n$ para algunas $z \in \mathbb{Z}$ y $n \in \mathbb{N}$, entonces existen $x,y \in \mathbb{Z}$ tales que $a = xn$ y $b = yn$.}
    Si $ab = z^n$ \\
    Por el Teorema Fundamental de la Aritmética sabemos:
    \begin{align}
        a &= \prod_i^{\infty} p_i^{\alpha_i}
        b &= \prod_i^{\infty} p_i^{\beta_i}
        z &= \prod_i^{\infty} p_i^{\gamma_i}
    \end{align}
    De forma que
    \begin{align}
        \prod_i^{\infty} p_i^{\alpha_i} \prod_i^{\infty} p_i^{\beta_i} &= \left(\prod_i^{\infty} p_i^{\alpha_i}\right)^n \\
        &= \prod_i^{\infty} p_i^{\alpha_im}
    \end{align}
    Por 8 y $(a;b)=1$\\
    $\forall \alpha_i \beta_i$ tal que $\min\{'alpha_i \beta_i\} = 0$ \\
    $\alpha_i = 0 \vee \beta_i = 0$ \\
    Además $\gamma_in = \alpha_i \beta_i$ \\
    $\therefore \alpha_in = \alpha_i \vee \alpha_1n = \beta_i$ \\
    $\vee \alpha_in = 0$ \\
    Notemos que si $\alpha_i \new 0, \beta_i = 0$ y si $\beta_i = 0 \rightarrow \alpha_i = \gamma_in$
    $\therefore \forall \alpha_i$ tal que $\alpha_i = \gamma_in \vee (\beta_i \neq 0 \wedge \alpha_i = 0)$ \\
    $\therefore \forall\alpha_i$ tal que $\alpha_i = k\gamma_in$ con $\displaystyle k = \{ \stackrel{1 \; si \; \beta_i = 0}{0 \; si \; \beta_i \neq 0} \} $ \\
    $\therefore a = \prod_i^{\infty} p_i^{\alpha_i^{k\gamma n}}$ con $\displaystyle k = \{ \stackrel{1 \; si \; \beta_i = 0}{0 \; si \; \beta_i \neq 0} \} $ \\
    $\therefore a = x^n, x \in \mathbb{Z}$ \\
    Analogamente para $b=y^n \qquad \blacksquare$
    