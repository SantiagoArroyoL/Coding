\section{a}
    \subsection{Calcula $\varphi(25)$ y $\varphi(4)$}
        \subsubsection{$\varphi(25)$}
            \begin{equation*}
                \begin{aligned}[c]
                    \notag (1;25)&=1 \\
                    \notag (2:25)&=1 \\
                    \notag (3:25)&=1 \\
                    \notag (4:25)&=1 \\
                    \notag (5:25)&=5 \\
                    \notag (6:25)&=1 \\
                    \notag (7:25)&=1 \\
                    \notag (8:25)&=1 \\
                    \notag (9:25)&=1 \\
                    \notag (10:25)&=5 \\
                    \notag (11:25)&=1 \\
                    \notag (12:25)&=1 \\
                    \end{aligned}
                    \qquad\qquad
                    \begin{aligned}[c]
                    \notag (13;25)&=1 \\
                    \notag (14;25)&=1 \\
                    \notag (15;25)&=5 \\
                    \notag (16;25)&=1 \\
                    \notag (17;25)&=1 \\
                    \notag (18;25)&=1 \\
                    \notag (19;25)&=1 \\
                    \notag (20;25)&=5 \\
                    \notag (21;25)&=1 \\
                    \notag (22;25)&=1 \\
                    \notag (23;25)&=1 \\
                    \notag (24;25)&=1 \\
                \end{aligned}
            \end{equation*}
            $\therefore \; \varphi(25) &= 20$
        \subsubsection{$\varphi(4)$}
        \begin{equation*}
            \begin{aligned}[c]
                \notag (1;4)&=1 \\
                \notag (2;4)&=7 \\
            \end{aligned}
            \qquad\qquad
            \begin{aligned}
                \notag (3;4) &= 1 \\
                \notag (2;4) &= 4
            \end{aligned}
        \end{equation*}
        $\therefore \; \varphi(4) &= 3$
    \subsection{Demuestra que $ 3^{20} \stackrel{25}{\equiv}1$ y $ 3^{20} \stackrel{4}{\equiv}1$}
        Por 21.1 sabemos que $\varphi(4) = 3 \wedge \varphi(25)= 20$
        \begin{proof}
            Como $3 \nmid 25$
            \begin{align}
               \tag{Por Teorema de Euler} 3^{20} \stackrel{25}{\equiv} 1
            \end{align}
            Como $3 \nmid 4$
            \begin{align}
               \tag{Por Teorema de Euler} 3^{2} &\stackrel{4}{\equiv} 1 \\
               \notag (3^2)^{10} &\stackrel{4}{\equiv} 1^{10} \\
               \notag 3^20 &\stackrel{4}{\equiv} 1^{10} \equiv 1 \\
               \notag \therefore \; 3^{20} &\stackrel{4}{\equiv} 1
            \end{align}
            Quedan demostradas ambas congruencias $\qquad \blacksquare$
        \end{proof}
    \subsection{Demuestra que $100 | 3^{20} −1$}
        De 21.2 sabemos que $3^{20} &\stackrel{25}{\equiv} 1 \wedge 3^{20} &\stackrel{4}{\equiv} 1$ \\
        Como $(25;4) = 1$ entonces $3^{20} \stackrel{100}{\equiv} 1$ \\
        $\therefore 100|3^{20}-1 \qquad \blacksquare$
    \subsection{¿Cuáles son los dos últimos dígitos de $3^{400}$ en base $10$?}
        Como de 21.3 sabemos que $3^{20} \stackrel{100}{\equiv} 1$ seguimos que: \\
        \begin{align}
            \notag \left(3^{20}\right)^{20} &\equiv 1^20 \\
            \notag 3^{400} &\stackrel{100}{\equiv} 1 \\
            \notag 3^{400}-1 &= 100x \\
            \notag 3^400 &= 100x+1
        \end{align}
        $\therefore$ Los últimos dos digitos son $01$