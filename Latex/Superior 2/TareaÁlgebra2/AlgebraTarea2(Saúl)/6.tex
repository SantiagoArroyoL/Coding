\section{Decimos que un entero de la forma $2^2^k + 1$ con $k \in \mathbb{N}$ es un número de Fermat. Las siguientes afirmaciones relacionan estos números con los primos:}
     \subsection{Todo entero positivo $n$ se expresa de manera única como $n = 2mx$ para $x \in \mathbb{Z}$ impar.}
    \begin{proof}
        Demostramos la existencia de un numero impar por inducción sobre $n$ \\
        Caso base $n = 1$\\
        \begin{align}
            \notag n = 1 = 1(1) = 2^0 \cdot 1
        \end{align}
        Paso Inductivo \\
        Supongamos $n > 1$, y que se cumple la afirmación  de que existe un numero impar tal que para todos los numeros menores a n se pueden expresar como $2^mx$, con x impar \\
        Por algoritmo de la división: \\
        $\exists ! q,r \in \mathbb{Z} | n = 2q+r, 0 \leq r \leq |2| = 2$
        \begin{align}
            \notag Si \; r &= 1\\
            \notag r &= 2a+1 \\
            \notag &= 2^0(2q+1)
        \end{align}
        Como $2q+1$ es impar terminamos
         \begin{align}
            \notag Si \; r &= 0\\
            \notag n &= 2q
        \end{align}
        Como $n>2q>q$ \\
        Por HI $q=2^mx$, con $x$ impar \\
        \begin{align}
            \notag \therefore n &= 2(2^mx) \\
            \tag{Con $x$ impar} n = 2^{m+1}x
        \end{align}
        Como se cumple el Caso Base y el Paso inductivo, se cumple $\forall n \in \mathbb{N}$, la afirmacion de que cualquier n se puede expresar como $2^mx$, con x impar \\ \\
        Ahora demostraremos la unicidad: \\
        Sean $m_1,m_2,x_1,x_2$ tales que $n = 2^{m_1}x_1$ y $n= 2^{m_2}x_2$ con $0 \leq m_1 \vee m_2$ y $x_1,x_2$ enteros impares positivos \\
        Supongamos que $m_1 \leq m_2 \rightarrow 0 \leq (m_2-m_1)$ \\
        Tenemos entonces que $2^{m_1}x_1 = 2^{m_2}x_2$ 
        \begin{align}
            \notag x_1 &= \frac{2^{m_2}x_2}{2^{m_1}} = 2^{(m_2-m_1)}x_2
        \end{align}
        Por hipotesis $x_1$ es un entero impar \\
        Entonces $ 2^{(m_2-m_1)}x_2$ es impar. \\
        \therefore $2^{m_2-m_1}$ y $x_2$ deben ser impares \\
        Sabemos además que $\forall z \in \mathbb{Z} $ tal que $z \neq 0 2^z$ es un número par excepto en un único caso\\
        Caso cuando $z=0$ tal que $2^0 = 1$, que es impar.\\
        Luego $2^{m_2-m_1} = 2^0 = 1$ lo que implicaría que $m_1=m_2$ \\
        Entonces $x_1 = 2^{m_2-m_1}x_2 = 1\cdot x_2 = x_2$ \\
        Es claro que si $m_1=m_2 \rightarrow x_1=x_2$ \\
        $\therefore$ Sólo existe una única expresión de esta forma para toda $n$ \\
    \end{proof}
    \\ Como se cumple la existencia y la unicidad, entonces demostramos que todo entero positivo $n$ se expresa de manera única como $n = 2mx$ para $x \in \mathbb{Z}$ impar. $\qquad \blacksquare$
    \subsection{Si $n \in \mathbb{N}$ es tal que $2^n + 1$ es primo, entonces $2^n + 1$ es un número de Fermat} 
    \begin{proof}
        Si $n \in \mathbb{N}$ tal que $2^n+1$ es primo. Como $ n \in \mathbb{N}$, por 6.1 tenemos:
        \begin{align}
            \tag{Con $x$ único e impar} (2^{2^mx})&+1 \\
            \notag (2^{2^mx})&+1^x \\  
            \notag 2^{2^mx}&-(-1)^x \\
            \tag{Por $Hint$} 2^{2^m}-(-1) &| 2^{2^mx} -(-1) = 2^{2^mx}+1 \\
            \notag \therefore 2^{2^m}+1 &| 2^{2^mx}+1 \\
            \tag{Por ser primo} 2^{2^m} + 1 &= 2^{2^mx}+1
        \end{align}
        Como $2^n+1= 2^{2^m} + 1$ queda demostrado que es un número de Fermat $\qquad \blacksquare$
    \end{proof}