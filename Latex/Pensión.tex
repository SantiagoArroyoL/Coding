\documentclass[13pt,letterpaper]{article}
\usepackage[margin=0.5in]{geometry}

\title{Presupuesto para la pensión}
\date{30-09-2019}
\author{Arroyo Lozano Santiago}

\begin{document}
  \maketitle
  \pagenumbering{arabic}

  (Todos los Totales son mensuales)
  \section{Gastos Mensuales Esenciales:}
  \subsection{Servicios}
    \begin{itemize}
      \item Télefono: 439 MXN
      \item Luz: 300 MXN
      \item Agua: 300 MXN
      \item Gas: 1000 MXN
    \end{itemize}
  \subsection{Súper(Mensual)}
    Se compra comida, consumibles que falten como desodorante, shampoo, plumas, cuadernos, la arena y croquetas del gato, etc. \\
    En ocasiones ropa y lo que se necesite, por lo mismo no se puede decir con exactitud cuánto se gasta al mes, pero un promedio sería de 3000 pesos
    \subsection{Leo}
    A Leo se le dan 350 pesos al día, va tres días a la casa de Liorna y uno a casa de Tita, al mes son 5600 pesos. \\
    Leo además pide 300 pesos semanales para comida y lo que se necesite pagar de improvisto \\ \\
  \textbf{Total:} 10989 MXN
  \section{Gastos Santiago:}
  \subsection{Dentista}
    La consulta es mensual debido al tratamiento, sin embargo Juan Castro dijo que en Diciembre me quitan los brackets.
    \subsection{Domingos}
      Recibo 500 pesos cada domingo, pero para facilitar todo llegan mensualmente en un pago de 2000 pesos.
    \subsection{Télefono:}
      Mi plan de télefono celular (MAX 5000) actualmente cuesta 549  pesos, pero como estoy pagando el equipo son 289 pesos extra. Dando un total de 838 pesos mensuales \\ \\
      \textbf{Total:} 4088 MXN
    \section{Total:}
      Sumando los demás totales tenemos que al mes es un gasto de:\\ \\
      {\Large \textbf{Total: 15077 MXN}} (Pesos Mensuales)
\end{document}
