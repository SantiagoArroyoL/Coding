\documentclass[12pt]{article}
%\usepackage{latin1}{inputenc}
%\usepackage{spanish}{babel}
\usepackage{graphicx}

\begin{document}

  \maketitle
  \title{Introducción a \LaTeX{}}
  \author{Santiago Arroyo Lozano}
  \date{14/Agosto/2019}
  %Ayuda por qué no funciona?


  \section{Introducción}
  Encontrar una estrategia ganadora para el juego del NIM.
  Se trata de un juego entre dos personas que realizan de
  manera alternada sus jugadas. Se tienen 11 palitos. En su
  turno, cada jugador puede tomar 1, 2 o 3 palitos, seg´un
  desee. Pierde quien se queda con el ´ultimo palito.

  \section{Estrategia}
  La estrategia consiste en dejar al contricante con 5 palitos sobre la mesa, ya que sin importar cuántos palitos tire después no podrá ganar la partida. Para conseguir eso se debe empezar removiendo 2 palitos
  Siendo Alpha nuestro turno y Beta el del contrincante:

  \begin{equation}
    \label{paso1}
    \alpha = 11 - 2 = 9
  \end{equation}

  Sin importar cuál sea el siguiente movimiento del contricante nosotros debemos asegurar que la próxima jugada dejará 5 palitos.
  Por ejemplo, suponiendo que el contrincante tirara 2:

  \begin{equation}
    \label{paso2}
    \beta = 9 - 2 = 7
  \end{equation}

  Para terminar sólo debemos disminuir la cantidad a 5:

  \begin{equation}
    \label{paso3}
    \alpha = 7 - 2 = 5
  \end{equation}

  No hay manera de que el contricante gane después de este tiro. Tal y como se muestran en los siguientes ejemplos:

  \begin{equation}
    \label{paso4a}
    \beta = 5 - 1 = 4
  \end{equation}
  \begin{equation}
    \label{paso5a}
    \alpha = 4 - 3 = 1
  \end{equation}
  GANAMOS!
  \begin{equation}
    \label{paso4b}
    \beta = 5 - 2 = 3
  \end{equation}
  \begin{equation}
    \label{paso5b}
    \alpha = 3 - 2 = 1
  \end{equation}
  GANAMOS!
  \begin{equation}
    \label{paso4c}
    \beta = 5 - 3 = 2
  \end{equation}
  \begin{equation}
    \label{paso5c}
    \alpha = 2 - 1 = 1
  \end{equation}
  GANAMOS!

  \section{Conclusión}
  Concluímos que empezar retirando 2 palitos para consecuentemente dejar 5 palitos en la mesa. Esta estrategia no es afectada por ninguna de las decisiones del contrincante, sin embargo el contrincante puede aplicar la misma estrategia y ganar el juego.

\end{document}
