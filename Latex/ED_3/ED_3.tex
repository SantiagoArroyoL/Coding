\documentclass[11pt,letterpaper]{article}
\usepackage[margin=0.5in]{geometry}
\usepackage[utf8]{inputenc}
\usepackage[T1]{fontenc}
\usepackage{amssymb}
\usepackage{amsthm}
\usepackage{amsmath}

\counterwithin*{equation}{section}
\renewcommand\qedsymbol{$\blacksquare$}

\title{Estructuras Discretas \\ Tarea 3}
\author{Arroyo Lozano Santiago \\ Becerril Lara Francisco Javier \\ García Chavelas Jonás }
\date{25 Octubre 2019}

\begin{document}

   \maketitle
   %\newpage

   \section{Indución matemática}
      \subsection{Demostrar $1^{2}+2^{2}+...+n^{2} = \frac{n(n+1)(2n+1)}{6}$}
         \hspace{1cm} Caso base $(n = 0)$ \\
         \begin{equation}\notag 0^{2} = \frac{0(0+1)}{6} \end{equation}
         \begin{equation}\notag 0 = 0 \end{equation}
         $\therefore$ Se cumple el Caso Base \\ \\
         $H.I. \; 1^{2}+2^{2}+...+n^{2} = \frac{n(n+1)(2n+1)}{6}$ \\
         $P.I. \; 1^{2}+2^{2}+...+n^{2}+(n+1)^{2} = \frac{(n+1)(n+1+1)(2(n+1)+1)}{6}$ \\
         \begin{equation}\tag{Por $H.I.$} = \frac{n(n+1)(2n+1)}{6} + \frac{6(n+1)^2}{6} \end{equation}
         \begin{equation}\tag{Desarrollamos} = \frac{n(n+1)(2n+1)+6(n+1)(n+1)}{6} \end{equation}
         \begin{equation}\tag{Factor Común} = \frac{(n+1)(n(2n+1)+6(n+1))}{6} \end{equation}
         \begin{equation}\tag{Desarrollamos} = \frac{(n+1)(2^{2}+7n+6)}{6} \end{equation}
         \begin{equation}\tag{Factor Común} =  2^{2}+7n+6 = \frac{(2n+4)(2n+3)}{2} \end{equation}
         \begin{equation}\tag{Dividimos} = \frac{(2n+4)(2n+3)}{2} = (n+2)(2n+3) \end{equation}
         \begin{equation}\tag{Desarrollamos} = \frac{(n+1)(n+2)(2n+3)}{6} \end{equation}
         \begin{equation}\tag{Conclusión} \therefore 1^{2}+2^{2}+...+n^{2}+(n+1)^{2} = \frac{(n+1)(n+1+1)(2(n+1)+1)}{6} \end{equation}
         Como se cumple el Caso Base y el $P.I.$ Queda demostrado.  \qedsymbol
      \subsection{Para $n$ personas hay $\frac{n(n-1)}{2}$ saludos}
         Inducción sobre $n$ \\
         Sabemos que cada persona que llega a la fiesta saluda a todos menos a si misma, entonces:\\
         $P.D. \; 1+2+...+n-1 = \frac{n(n-1)}{2}$ \\
         \hspace{1cm} Caso Base $(n = 2)$
         \begin{equation}\notag 2-1 = \frac{2(2-1)}{2}\end{equation}
         \begin{equation}\notag 1 = \frac{2}{2} = 1\end{equation}
         $\therefore$ Se cumple el Caso Base \\ \\
         $H.I. \; 1+2+3+...+n-1 = \frac{n(n-1)}{2}$ \\
         $P.I. \; 1+2+...+n-1+(n+1)-1 = \frac{n+1(n+1-1)}{2}$ \\
         \begin{equation}
            1+2+...+n-1+(n+1)-1 = 1+2+...+n-1+n
         \end{equation}
         \begin{equation}\tag{Por $H.I.$}
            = \frac{n(n-1)}{2}+n
         \end{equation}
         \begin{equation}\tag{Sumamos $n$} = \frac{n(n-1)+2n}{2} \end{equation}
         \begin{equation}\tag{Desarrollamos} = \frac{n^{2}-n+2n}{2} = \frac{n^{2}+n}{2}\end{equation}
         \begin{equation}\tag{Conmutatividad} = \frac{n(n+1)}{2} = \frac{(n+1)n}{2} \end{equation}
         \begin{center}Sumamos y restamos 1, manteniendo la igualdad\end{center}
         \begin{equation} = \frac{(n+1)(n+1-1)}{2} \end{equation}
         \begin{equation}\tag{Conclusión} \therefore 1+2+...+n-1+(n+1)-1 = \frac{(n+1)(n+1-1)}{2}\end{equation}
         Como se cumple el Caso Base y el $P.I.$ Queda demostrado.  \qedsymbol
         \subsection{Considera $n$ fórmulas de la lógica proposicional $\varphi_{1},\varphi_{2}...\varphi_{n}$}
            Inducción sobre $n$ \\
            \hspace{1cm} Caso Base $(n = 1)$ \\
            $\rightarrow$ Como $\forall_{i} \in $ \textbraceleft $ 1 $\textbraceright$ \mathcal{I}(\varphi_{i}) = 1$, entonces $\mathcal{I}(\varphi_{i}) = 1$ \\
            $\leftarrow$ Como $ \mathcal{I}(\varphi_{i}) = 1$ y $1 \in $  \textbraceleft $ 1 $\textbraceright, entonces $1 = i$, tal que $\mathcal{I}(\varphi_{i}) = 1$ \\
            $\therefore$ Se cumple el Caso Base \\ \\
            $H.I. \; \mathcal{I} (\varphi_{1} \wedge \varphi_{2}\wedge...\wedge \varphi_{n}) = 1$\\
            $P.I. \; \mathcal{I} (\varphi_{1} \wedge \varphi_{2}\wedge...\wedge \varphi_{n}\wedge \varphi_{n+1}) = 1 \leftrightarrow \forall_{i} \in$ \textbraceleft $1,...,n,n+1$ \textbraceright, $ \mathcal{I}(\varphi_{i}) = 1$ \\
            \begin{align}\tag{Por $H.I.$}
               Sabemos \; que \; \mathcal{I} (\varphi_{1} \wedge \varphi_{2}\wedge...\wedge \varphi_{n}) = 1 \\
               es \; decir, \; \mathcal{I} (\varphi_{1} \wedge \varphi_{2}\wedge...\wedge \varphi_{n}) = 1 = \top, entonces \\
               \top \vee \mathcal{I}(\varphi_{i}) = 1.
            \end{align}
            Como deducimos que esta disyunción es cierta, entonces $ \mathcal{I}(\varphi_{n+i}) = 1$ y como $ \mathcal{I}(\varphi_{i}) = 1$, entonces $n+1 = i \in $\textbraceleft $ 1,...,n,n+1 $\textbraceright \\
            Como se cumple el Caso Base y el $P.I.$ Queda demostrado.  \qedsymbol
         % Recursión*********************
   \section{Recursión}
   Sea a una hoja del arbol binario y t1 y t2 arboles binarios

      \subsection{Preorden}
      \begin{ttfamily}
         preorden:: ArbolBinario -> [a] \\
         preorden NIL = [] \\
         preorden t1 a t2 = [a] ++ (preorden t1 ++ preorden t2)
      \end{ttfamily}

      \subsection{Inorden}
      \begin{ttfamily}
         postorden:: ArbolBinario -> [a]\\
         postorden NIL = [] \\
         postorden t1 a t2 = (postorden t1 ++ postorden t2 ++ [a])
      \end{ttfamily}

      \subsection{Postorden}
         \begin{ttfamily}
            inorden:: ArbolBinario -> [a] \\
            inorden NIL = [] \\
            inorden t1 a t2 = (inorden t1 ++ [a] ++ inorden t2)
         \end{ttfamily}

      \subsection{Hanoi}
      Para definir la función recursiva del juego Hanoi debemos primero definir la palabra Hanoi.\\
      "Hacer el Hanoi" se refiere a mover toda la pirámide de $n$ piezas a su destino final.\\
      Para lograrlo hacemos el hanoi de $n - 1$ (Primer hanoi) para liberar la pieza n y moverla hasta la última columna (1 movimiento). \\
      Luego el hanoi de $n-1$ una última vez para acomodarlo encima de $n$ (Segundo hanoi).\\
      Hicimos 2 veces el hanoi de $n - 1$ más sólo un movimiento de la pieza n.\\
      Y nuestro caso base sería 0, porque con 0 piezas, 0 movimientos, lo que nos da:\\
         \begin{ttfamily}
            hanoi:: Int -> Int \\
            hanoi 0 = 0 \\
            hanoi n = 2 * hanoi (n-1) + 1
         \end{ttfamily}
      \subsection{Sea $ S $ = \textbraceleft $ w $ el conjunto de cadenas 1 y 0 | tiene número par de ceros \textbraceright}

         \begin{itemize}
            \item $0 \in S $, $1  \in S $
            \item Si $ a \in S $, entonces $a$ tiene $n | n \; sea \; par \in \mathbb{N}$ cantidad de $0$ concatenados por cualquier lado
            \item Si $ b \in S $, entonces $b$ tiene $n | n \in \mathbb{N}$ cantidades de $1$ concatenados por cualquier lado
            \item Si $ c \in S $, $c = 0b0 \vee ba \vee ab \vee cc \vee ccc$
            \item Éstas y sólo éstas son expresiones de $ S $
         \end{itemize}

         \subsubsection{Principo de Inducción estructural para el conjunto $ S $}
         Sea que el conjunto $ S $ tiene número par de ceros, para demostrar que $ w $ tenga número par de ceros para cada $w \in S$, es suficiente probar $P(a)$
   %INDUCCIÓN ****************
   \section{Inducción estructural}
      \subsubsection{rev(xs ++ ys) = rev(ys) ++ rev(xs)}
         Procederemos a demostrar por Induccion\\
         Induccion sobre xs.\\
         a) Caso Base. $xs = [\ ]$\\ \\
         $rev ([\ ] + + ys) = rev(ys + + rev([\ ])$  \hspace{1cm} (Por definicion de reversa.) \\
         $rev(ys) = rev(ys) + + [\ ]$\\
         $rev(ys) = rev(ys)$\\ \\
         $\therefore$ Se cumple el Caso Base \\ \\
         b) Hipotesis de Induccion. $ rev(xs ++ ys) = rev(ys) + + rev(xs)$\\ \\
         c) Paso Inductivo. Por Demostrar $rev(a:xs ++ ys) = rev(ys) + + rev(a:xs)$ \\ \\
         $rev(ys) + + rev(a:xs) = rev(ys) ++ rev(xs) + + [a]$ \hspace{2cm} (Por definicion de reversa)\\
         $ = rev(xs + + ys) + + [a]$ \hspace{7cm} (Por Hipotesis de Induccion)\\
         $ = rev(a:xs + + ys)$ \hspace{7cm} (Por conmutatividad y definicion de reversa)\\
         $\therefore$ Se cumple el principio de Induccion Estructural $ \blacksquare$
      \subsubsection{rev(rev(l)) = l}
         Procederemos a demostrar por Induccion\\
         Induccion sobre l.\\
         a) Caso Base. $l = [\ ]$\\ \\
         $rev (rev[\ ]) = rev([\ ])$ \hspace{2cm}( Por definicion de reversa.) \\
         $[\ ] = [\ ]$ \hspace{2cm} (Por definicion de reversa.)\\ \\
         $\therefore$ Por lo tanto se cumple el Caso Base \\ \\
         b) Hipotesis de Induccion. $ rev(rev(l)) =l$\\ \\
         c) Paso Inductivo. Por Demostrar $ rev(rev(a:l)) = a:l$\\ \\
         $ rev(rev(a:l)) = rev(rev(l) + + a)$ \hspace{1cm} (Por definicion de reversa)\\
         $ = l ++ a = a:l$ \hspace{3cm} (Por Hipotesis de Induccion y concatenacion)·\\ \\
         $\therefore$ Se cumple el principio de Induccion Estructural $ \blacksquare$
      \subsubsection{desorden(T) = rev(preorden(T))}
         Procederemos a demostrar por Induccion\\
         Induccion sobre T.\\
         a) Caso Base. $T = NIL$\\ \\
         $desorden(NIL) = rev(preorden(NIL))$ \\
         $[\ ] = rev([\ ])$ \hspace{1cm} (Por definicion de desorden)\\
         $[\ ] = [\ ]$ \hspace{1cm} (Por definicion de reversa)\\ \\
         $\therefore$ Se cumple el Caso Base \\ \\
         b) Hipotesis de Induccion $desorden((T)) = reversa(preorden(T))$ \\ \\
         c)Paso Inductivo. \\
         Por Demostrar $desorden(mktree(T_{i}, T_{R}, NIL)) = reversa(preorden(T_{i}, T_{R}, NIL))$ \\ \\
         $ =  desorden(NIL) + + desorden(T_{i}) + + [T_{R}]$ \hspace{1cm} (Definicion de desorden.) \\
         $ = [\ ] + + reversa(preorden(T_{i})) + + [T_{R}]$ \hspace{1cm} (Por Hipotesis de Induccion.) \\
         $ = reversa([T_{i}]   + + [\ ] + + [\ ]) + + [T_{R}]$ \\
         $ = reversa([T_{R}]:[T_{i}] + + [\ ])$
         $ = reversa(preorden(T_{i},T_{R},NIL))$ \\ \\
         $\therefore$ Por lo tanto se cumple el principio de Induccion Estructural $ \blacksquare$
      \subsubsection{$2\times$sum(l) = sum(doble(l))}
         Demostración por Principio de Inducción sobre $ l $.\\
          Caso base ($ l = [\,]$)\\
         $ 2 \times sum([\,]) = sum(doble([\,])) $ \\
         $ \qquad 0 = sum([\,]) $ \\
         $ 0 = 0 $ \\ \\
         $ \therefore $ Se cumple el caso base.
         Hipótesis de inducción: $ 2 \times sum(l) = sum(doble(l)) $
         Paso inductivo: $ 2 \times sum(a:l) = sum(doble(a:l)) $ \\
         $ 2 \times(a:l) = 2 \times (a + sum(l)) $ \\
         Por otro lado: $ = 2 \times a + 2 \times sum(l) $ \\
         % $ sum(doble(a:l)) = sum((2 \time a):doble(l)) $ \\
         $ = (2 \times a) + sum(doble(l)) $ \\
         $ 2 \times a + 2 \times sum(l) = 2 \times a + 2 \times sum(l) $ \\ \\
         $ \therefore 2 \times sum(a:l) = sum(doble(a:l)) $ \\ \\
         $ \therefore $ Se cumple el paso inductivo.\\ \\
         $ \therefore $ Se cumple el principio de inducción en $ 2 \times sum(l) = sum(doble(l)) $. $ \blacksquare $
         \subsubsection{Toda fórmula de la lógica proposicional, se puede escribir utilizando únicamente conjunciones y negaciones.}
         \begin{enumerate}

         \item Toda fórmula de la lógica proposicional, se puede escribir
           utilizando únicamente conjunciones y negaciones. \\
           \emph{Demostración por Principio de Inducción sobre $ \varphi \, | \, \varphi $ es una fórmula de la lógica proposicional.}

           \begin{enumerate}

           \item Caso base ($ \varphi \equiv P $):

             $ \varphi \equiv P \equiv \neg \neg P \equiv (P \wedge P) $ \\
               $ \therefore $ se cumple el caso base.

           \end{enumerate}

         \item Hipótesis de inducción: \\
           Suponemos que existen $ P $ y $ Q $ tal que se pueden escribir \\
           utilizando únicamente conjunciones y negaciones.

         \item Paso inductivo:

           \begin{enumerate}

           \item ($ P $)\\
             $ P \equiv \neg \, \neg P $\\
             $ \therefore $ Se cumple el caso a).

           \item ($ \neg P $)\\
             $ \neg P $ \\
             $ \therefore $ La demostración es inmediata, puesto que se cumple por definición.

           \item ($ P \wedge Q $)\\
             $ P \wedge Q $ \\
             $ \therefore $ La demostración es inmediata, puesto que se cumple por definición.

           \item ($ P \vee Q $) \\
             $ (P \vee Q) \equiv \neg (\neg P \wedge \neg \, Q) $ \\
             $ \therefore $ Se cumple el caso d).

           \item ($ P \rightarrow Q $) \\
             $ (P \rightarrow Q) \equiv (\neg P \vee Q) \equiv \neg (P \wedge \neg \, Q) $ \\
             $ \therefore $ Se cumple el caso e).

           \item ($ P \longleftrightarrow Q $)\\
             $ (P \longleftrightarrow Q) \equiv (\neg P \vee Q) \wedge (\neg \, Q \vee P) $ \\
             $ \equiv \neg (P \wedge \neg \, Q) \wedge \neg (Q \wedge \neg \, P) $ \\
             $ \therefore $ Se cumple el caso f).

           \end{enumerate}

           $ \therefore $ Se cumple el paso inductivo. \\ \\
           $ \therefore $ Se cumple con el principio de inducción. \qedsymbol

         \end{enumerate}
\end{document}
