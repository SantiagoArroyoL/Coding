\documentclass[11pt,letterpaper]{article}
\begin{document}
    Hola.
    Tenemos una relación $\mathcal{R}$ \\

    $\mathcal{R}=(G,A,A) $ \textbraceleft donde  $A \neq \emptyset $ es un conjunto cualquiera y  \textbraceright $G\subset A\timesA$ \\

    En el caso que nos interesa \\

    $G=\emptyset$ luego se tiene para $a,b \in A$ \\

    1) a \textbraceleft $\mathcal{R}$ \textbraceright  a  \textbraceleft ya que \textbraceright (a,a)\not\in G \\
    2) a  $\mathcal{R}$ b  $\Rightarrow$ b  $\mathcal{R}$ a  \textbraceleft por ser el antecedente falso la implicacion es verdadera \textbraceright \\

    Las demás propiedades, las hallas de la misma manera.

\end{document}
